% !TEX encoding = UTF-8
% !TEX TS-program = pdflatex
% !TEX root = ../tesi.tex

%**************************************************************
\chapter{Conclusione}
\label{cap:conclusione}
%**************************************************************

%\intro{Breve introduzione al capitolo}\\

%**************************************************************
\section{Consuntivo finale}
Gli scostamenti rilevati nelle prime tre attività sono dovuti alla difficoltà di reperimento di un computer con sistema operativo MacOS l'analisi e la costruzione del \emph{\gls{pocg}} relativo a Siri. Perciò si è deciso di privilegiare lo studio di Assistant ed Alexa con ulteriori approfondimenti. \\
Nelle attività finali, invece, si sono verificati degli scostamenti perché è stato svolto un approfondimento sulla possibile utilità del motore di regole a scapito, in comune accordo con il tutor aziendale, a scapito del tempo dedicato alla documentazione. \\
Il consuntivo finale è quindi riportato nella seguente tabella.
%\begin{table}
	\begin{longtable}{|p{4.5cm}|c|c|c|}
		\hline
		\textbf{Attività} & \textbf{Ore pianificate} & \textbf{Ore effettive} & \textbf{Scostamento} \\
		\hline
		Studio di Assistant e implementazione di un \emph{\gls{pocg}}. & 40 & 48 & +8 \\
		\hline
		Studio di Alexa e implementazione di un \emph{\gls{pocg}}. & 40 & 48 & +8 \\
		\hline
		Studio di Siri e implementazione di un \emph{\gls{pocg}}. & 40 & 32 & -8 \\
		\hline
		Test e documentazione comparativa di quanto svolto nelle settimane precedenti. & 40 & 40 & 0 \\
		\hline
		Apprendimento della tecnologia Zucchetti per il riconoscimento e l'elaborazione di comandi vocali. & 40 & 32 & -8 \\
		\hline
		Realizzazione di un'applicazione che implementi una \emph{\gls{nlug}} basata su una \emph{\gls{gramg}} costruita mediante la tecnologia di Zucchetti. & 40 & 40 & 0 \\
		\hline
		Implementazione della capacità conversazionale con scambio e memorizzazione di informazioni. & 40 & 48 & +8 \\
		\hline
		Test e documentazione di quanto svolto nelle settimane precedenti. & 40 & 32 & -8 \\	
		\hline
		\caption{Consuntivo finale}
	\end{longtable}
%\end{table}
%\pagebreak
%**************************************************************
\section{Raggiungimento obiettivi}
Il raggiungimento degli obiettivi fa riferimento alla loro pianificazione descritta nella sezione §\hyperref[obiettivi]{2.2}.
\subsection{Obiettivi obbligatori}
\begin{itemize}
	\item \textbf{OB-1}: obiettivo raggiunto. Inizialmente è stata svolta un'analisi preliminare di tutte le capacità di Assistant e successivamente, sulla base di indicazioni e preferenze del tutor aziendale, sono state approfondite alcune singole funzionalità;
	\item \textbf{OB-2}: obiettivo raggiunto. Durante l'analisi delle capacità di Assistant è stato deciso di implementare un \emph{\gls{pocg}} legato alle App Actions;
	\item \textbf{OB-3}: obiettivo raggiunto. Inizialmente è stata svolta un'analisi preliminare di tutte le capacità di Alexa e successivamente, sulla base di indicazioni e preferenze del tutor aziendale, sono state approfondite alcune singole funzionalità;
	\item \textbf{OB-4}: obiettivo raggiunto. Durante l'analisi delle capacità di Alexa è stato deciso di implementare un \emph{\gls{pocg}} legato alle Skill con capacità conversazionale;
	\item \textbf{OB-5}: obiettivo raggiunto. È stato redatto un documento che riporta un'analisi dettagliata e comparativa degli assistenti virtuali studiati;
	\item \textbf{OB-6}: obiettivo raggiunto. Sono state studiate regole e caratteristiche della tecnologia Zucchetti per costruire \emph{grammatiche} che riconoscono il linguaggio naturale;
	\item \textbf{OB-7}: obiettivo raggiunto. In comune accordo con il tutor aziendale è stata realizzata un'applicazione con una propria \emph{\gls{nlug}} basata su una \emph{\gls{gramg}} generata tramite la tecnologia Zucchetti. Il suo scopo principale è riconoscere ed elaborare la data di nascita e di compleanno dell'utente interagendo mediante un'interfaccia vocale.
\end{itemize}
\subsection{Obiettivi desiderabili}
\begin{itemize}
	\item \textbf{OD-1}: obiettivo raggiunto. Implementazione della capacità conversazionale mirata a reperire e memorizzare tutti i dati relativi a data di nascita e di compleanno dell'utente per soddisfare lo scopo dell'applicazione.
\end{itemize}
\subsection{Obiettivi facoltativi}
\begin{itemize}
	\item \textbf{OF-1}: obiettivo raggiunto. Inizialmente è stata fatta un'analisi preliminare di tutte le capacità di Siri e successivamente, sulla base di indicazioni e preferenze del tutor aziendale, sono state approfondite le singole funzionalità;
	\item \textbf{OF-2}: obiettivo non raggiunto. Durante l'analisi delle capacità di Siri è stato deciso di implementare un \emph{\gls{pocg}} che permetta l'utilizzo delle Shortcuts. Tuttavia a causa delle restrizioni dell'account sviluppatore Apple a disposizione non si è potuta portare a compimento tale attività.
\end{itemize}
\subsection{Tabella riassuntiva}
%\begin{table}
	\begin{longtable}{|c|p{8.5cm}|c|}
		\hline
		\textbf{Codice} & \textbf{Obiettivo} & \textbf{Esito} \\
		\hline
		OB-1 & Studio e analisi delle capacità e delle modalità di utilizzo lato sviluppatore di Assistant. & Raggiunto \\
		\hline
		OB-2 & Implementazione di un \emph{\gls{pocg}} che realizzi una funzionalità di Assistant accordata sulla base dei risultati della ricerca. & Raggiunto \\
		\hline
		OB-3 & Studio e analisi delle capacità e delle modalità di utilizzo lato sviluppatore di Alexa. & Raggiunto \\
		\hline
		OB-4 & Implementazione di un \emph{\gls{pocg}} che realizzi una funzionalità di Alexa accordata sulla base dei risultati della ricerca. & Raggiunto \\
		\hline
		OB-5 & Redazione di un documento che riporta un'analisi dettagliata e comparativa degli assistenti virtuali studiati. & Raggiunto \\
		\hline
		OB-6 & Studio di regole e caratteristiche dell'algoritmo di Zucchetti per la costruzione di \emph{grammatiche} che riconoscono il linguaggio naturale. & Raggiunto \\
		\hline
		OB-7 & Realizzazione di un'applicazione con una propria \emph{\gls{nlug}} basata su una \emph{\gls{gramg}} generata mediante la tecnologia Zucchetti che interagisca con gli utenti tramite interfaccia vocale. & Raggiunto \\
		\hline
		OD-1 & Implementazione della capacità conversazionale tramite lo scambio di informazioni specifiche, possibilmente memorizzate, durante la conversazione mirato a soddisfare una determinata funzionalità. & Raggiunto \\	
		\hline
		OF-1 & Studio e analisi delle capacità e delle modalità di utilizzo lato sviluppatore di Siri. & Raggiunto \\	
		\hline
		OF-2 & Implementazione di un \emph{\gls{pocg}} che realizzi una funzionalità di Siri accordata sulla base dei risultati della ricerca. & Non raggiunto\\	
		\hline
		\caption{Raggiuntimento degli obiettivi}
	\end{longtable}
%\end{table}
%\pagebreak
%**************************************************************
\section{Valutazione personale}
\subsection{Conoscenze acquisite}
Durante l'attività di stage ho acquisto numerose conoscenze affrontando argomenti non presenti nel mio piano di studi universitario e ampliate altre che avevo appreso negli anni precedenti. Esse sono riportate nel seguente elenco:
\begin{itemize}
	\item principi e funzionalità degli assistenti virtuali: ho scoperto molte funzionalità degli assistenti virtuali che ho riportato nel capitolo  §\hyperref[cap:descrizione-stage]{3} ma soprattutto ho appreso molte nozioni sul loro funzionamento e su come possono essere utili sia agli utenti nella loro vita quotidiana sia alle azienda nella realizzazione dei loro prodotti. Le due principali sono il meccanismo degli intenti che rappresenta un nuovo modo di gestire l'interazione con gli utenti e l'insieme dei principi di realizzazione dell'interfaccia vocale che rappresentano un nuovo modo di interagire con gli utenti ancora poco esplorato ma ricco di potenzialità. Questi sono stati inoltre sperimentati nell'applicazione costruita;
	\item costruzione di applicazioni Android: durante lo sviluppo del \emph{\gls{pocg}} relativo ad Assistant ho imparato da autodidatta le basi della costruzione di un'applicazione Android con il linguaggio Kotlin e come attivarne le funzionalità tramite Assistant;
	\item costruzione di applicazioni in linguaggio Swift per l'ecosistema Apple: durante lo sviluppo del \emph{\gls{pocg}} relativo a Siri ho imparato da autodidatta le basi della costruzione di un'applicazione per iOS in linguaggio Swift e come predisporre delle Shortcuts personalizzabili dall'utente e attivabili tramite Siri;
	\item utilizzo di Javascript in ambiti nuovi: lo sviluppo dell'applicazione, a parte l'interfaccia grafica, è interamente realizzato in Javascript. Ho quindi imparato a realizzare software scritti in questo linguaggio per scopi diversi da quelli presentati durante il mio percorso di studi ed inoltre questo mi ha fatto capire la grande versatilità e l'ampio utilizzo che ne viene fatto nella costruzione di applicazioni;
	\item comprensione del linguaggio naturale: la tecnologia Zucchetti per l'interpretazione del linguaggio naturale si basa sull'utilizzo di \emph{grammatiche} per le quali ho avuto una formazione teorica durante il mio percorso di studi ma grazie a questa esperienza ho imparato ad implementarle. Ho inoltre appreso i principi di realizzazione di una \emph{\gls{nlug}}, oltre a metodologie per l'elaborazione dei risultati, e delle numerose problematiche e difficoltà che ne derivano;
	\item capacità di utilizzare nuovi strumenti: la varietà di tecnologie affrontate mi ha portato ad imparare l'utilizzo di altrettanti strumenti ausiliari quali Xcode, AndroidStudio e i diagrammi \emph{\gls{rldg}}.
\end{itemize}
\subsection{Competenze acquisite}
Grazie all'esperienza di stage, oltre alle conoscenze, ho acquisito nuove competenze che mi hanno permesso di maturale molto a livello professionale. Esse sono riportate nel seguente elenco:
\begin{itemize}
	\item versatilità nell'utilizzo e nell'apprendimento di nuove tecnologie: la maggior parte dello stage è stato centrata su analisi e apprendimento di molte tecnologie, talvolta diverse tra loro. Ciò mi ha permesso di migliorare nell'approccio alla ricerca, nell'autoapprendimento e nella capacità di fare paragoni definendo vantaggi e svantaggi;
	\item capacità di elaborare ragionamenti in ottica di innovazione: uno degli scopi principali delle mie ricerche è stato capire quali delle funzionalità e degli strumenti offerti fossero utili ai progetti aziendali. Esse infatti sono state riportate e discusse più volte con il mio tutor ed altri colleghi in sede di \emph{\gls{brainstorming}}\glsfirstoccur. Questo mi ha permesso contribuire ed imparare a fare dei ragionamenti mirati all'innovazione e al miglioramento a prodotti già esistenti o in via di sviluppo.
\end{itemize}
\subsection{Tecnologie e strumenti utilizzati}
Le tecnologie e gli strumenti con cui ho lavorato sono numerosi. Nella realizzazione dei \emph{\gls{pocg}} legati agli assistenti virtuali ho utilizzato Kotlin con Android Studio nell'applicazione per Assistant, Javascript con la console da sviluppatori nella Skill per Alexa e Swift con Xcode nell'applicazione per Siri. \\
Per l'applicazione che implementa la \emph{\gls{nlug}}, invece, ho utilizzato HTML e CSS per l'interfaccia grafica e Javascript versione \emph{\gls{es6}} per tutte le altre componenti mentre come strumento ho utilizzato WebStorm. \\
La maggior parte di queste tecnologie non sono state affrontate, oppure solo in modo marginale, durante il mio percorso di studi; tuttavia quella di cui ho avvertito maggior carenza è indubbiamente Javascript in quanto ne ho fatto largo uso in ambiti totalmente nuovi.
\subsection{Metodologia di lavoro}
Lo stage si è svolto in remoto e questa è stata una nuova esperienza sia per me che per il mio tutor. Per cercare di simulare nel modo migliore possibile la presenza in azienda siamo rimasti in contatto quotidianamente e, in aggiunta, ho redatto un registro riportando ogni giorno tutte le attività svolte. \\
Lo svantaggio riscontrato dal lavoro in remoto è la mancanza del rapporto diretto con il tutor ed i colleghi che, a mio parere, è formativo dal punto vista sia personale che professionale. Tuttavia porta con sé alcuni vantaggi quali maggior flessibilità negli orari e la non necessità di viaggiare in auto per recarsi in ufficio. Inoltre, soprattutto per lavori in ambito informatico, esistono numerosi strumenti che permettono di lavorare da casa in modo efficace ed efficiente annullando parte degli svantaggi. \\
Perciò è stata una sfida ed una possibilità di sperimentare una nuova modalità di lavoro che mi ha permesso ugualmente di raggiungere gli obiettivi prefissati con risultati notevoli.
\subsection{Analisi retrospettiva dei risultati}
Il lavoro svolto durante lo stage non è stato strutturato per realizzare un prodotto finito e pronto all'uso ma per eseguire una ricerca esplorativa su argomenti di interesse per i progetti aziendali, che trova concretezza in alcuni \emph{\gls{pocg}} dimostrativi. In seguito ho analizzato i risultati ottenuti e da essi sono emersi degli ottimi spunti di riflessione. \\
Il primo è: dall'interfaccia vocale l'utente si aspetta intelligenza. Questo l'ho percepito durante lo sviluppo dell'applicazione, quando ancora non c'era una copertura sufficiente nella comprensione dei comandi vocali. Infatti, mentre facevo eseguire dei test ad alcuni utenti esterni, spesso la \emph{\gls{nlug}} non riusciva a comprendere le frasi pronunciate facendoli spazientire; loro si aspettavano di colloquiare con un sistema intelligente al pari di una persona. Inoltre, se si considera l'aspetto conversazionale, il problema diventa ancora più accentuato in quanto l'utente si aspetta di interagire con un software che comprenda il contesto del dialogo in corso e abbia capacità di memoria. \\
Questa considerazione ha grande rilevanza perché evidenzia l'attenzione ai minimi dettagli che si deve prestare durante la progettazione dell'interfaccia vocale. Infatti, nonostante i grandi vantaggi che porta, quali la velocità e la comodità di utilizzo, presenta dei rischi notevoli sulla sua buona riuscita nei confronti degli utenti. \\
La seconda riflessione, invece, è la seguente: l'utilizzo dell'interfaccia vocale deve essere giustificato rispetto a quella grafica. Questa considerazione nasce dal fatto che l'interfaccia grafica è in assoluto la più diffusa, grazie anche ai dispositivi mobili, diventando lo standard per gli utenti. Il vantaggio dell'interfaccia vocale però risiede nella maggior comodità e rapidità legata all'esecuzione di operazioni semplici per le quali quindi risulta giustificata; tuttavia in attività complesse, possibilmente svolte in ambienti rumorosi, risulta assai svantaggiosa e lo sforzo per implementarla può non essere pienamente giustificato. \\
L'utilizzo di un'interfaccia vocale con una \emph{\gls{nlug}} rappresenta senza dubbio una tecnologia con grandi potenzialità ancora inesplorate ma dalle due precedenti riflessioni emergono i suoi limiti attuali. \\
La terza riflessione si propone come una possibile idea per risolvere i problemi riscontrati: costruire un'interfaccia ibrida che metta assieme gli aspetti positivi di quella vocale e di quella grafica. Più nello specifico mi riferisco alla costruzione di un'interfaccia che da un lato abbia una componente grafica per gli elementi più complessi e rilevanti in modo che l'utente non percepisca senso di smarrimento o difficoltà nell'utilizzo e dall'altro abbia una componente vocale per le operazioni più facili e di immediata esecuzione. In questo modo si eviterebbero buona parte dei problemi espressi in precedenza senza però sfruttarne a pieno le potenzialità. \\
La quarta ed ultima riflessione si stacca leggermente dalle precedenti poiché si sofferma sull'interfaccia vocale e sul possibile pattern architetturale di applicazioni che ne fanno uso. Questa riflessione è emersa in uno degli ultimi colloqui con il tutor aziendale in riferimento al pattern Model-View-Controller ed è la seguente: la View è per natura una componente passiva che esegue il rendering del modello e non possiede stato; tuttavia questo diventa impossibile da applicare con l'interfaccia vocale e la motivazione risiede nella capacità conversazionale che si vuole offrire all'utente. Più in dettaglio la View contiene sempre tutto quello che l'utente deve vedere mentre durante una conversazione il contesto si costruisce nel tempo rendendo impossibile presentare l'intero contenuto. Inoltre questo contesto, che necessariamente deve essere memorizzato, non appartiene propriamente al modello dell'applicazione in quanto non rappresenta nè i dati nè le operazioni da eseguire e nemmeno al Controller perché ha mansioni totalmente differenti. Perciò tali considerazioni portano a pensare che sia legati alla View. Da qui nasce l'idea che il pattern Model-View-Controller non sia direttamente applicabile a programmi basati su interfaccia vocale ma necessiti di una modifica nella View, trasformandola in un componente con capacità di memoria che mantiene il contesto della conversazione. \\
Queste sono le riflessioni esprimono l'esito conclusivo dell'analisi retrospettiva dei risultati riscontrati durante lo stage. \\
Infine io sono rimasto molto soddisfatto dagli argomenti trattati, dalla tipologia di lavoro svolto, dal rapporto lavorativo con il tutor aziendale ed in generale dall'intera esperienza.

%TODO: Impatto del lavoro sul progetto aziendale generale (il prodotto è utilizzato?), possibili punti di insoddisfazione e relativi miglioramenti ed estensioni
%TODO: VEDI FOGLIO RIASSUNTIVO ULTIMI RAGIONAMENTI SVOLTI
%todo: COMMENTI PERSONALI FINALI

