% !TEX encoding = UTF-8
% !TEX TS-program = pdflatex
% !TEX root = ../tesi.tex

%**************************************************************
\chapter{Lo stage}
\label{cap:lo-stage}
%**************************************************************

%\intro{Brevissima introduzione al capitolo}\\

%**************************************************************
\section{Descrizione del progetto}
Il progetto di stage è legato ad uno degli ambiti più innovativi della ricerca scientifica e informatica: il riconoscimento e l'elaborazione del linguaggio naturale da parte di un calcolatore. \\
Nel corso degli ultimi anni, l'azienda sta affrontando la sfida della comprensione di comandi vocali per permette agli utenti di comunicare con i propri prodotti. È infatti in via di sviluppo una loro tecnologia in grado di creare grammatiche atte a riconoscere il linguaggio naturale mediante regole ben precise. In merito a questo argomento si è costruito il progetto di stage articolato poi in due parti:
\begin{enumerate}
	\item studio degli assistenti virtuali più comuni ed utilizzati;
	\item creazione di un'applicazione sotto forma di \textit{proof of concept} capace di intrattenere una conversazione.
\end{enumerate}
La prima parte richiede un'analisi dettagliata e comparativa dei tre assistenti virtuali più utilizzati e, secondo diversi studi, migliori:
\begin{itemize}
	\item Assistant: assistente di Google presente in quasi tutti i dispositivi Android e nei prodotti della linea Google Home e Nest;
	\item Alexa: assistente di Amazon presente in tutta la linea di dispositivi Echo ed integrabile anche nei dispositivi con sistema operativo Android e IOS;
	\item Siri: assistente di Apple presente esclusivamente in tutti i prodotti Apple.
\end{itemize}
Lo studio deve comprendere tutte le capacità che posseggono, con particolare attenzione alla costruzione di \textit{Skill} personalizzate ed integrabili nei propri progetti, ad eccezione di quelle relative alla costruzione di \textit{smart home} in quanto non sono di diretto interesse per i prodotti aziendali; l'azienda infatti punta alla realizzazione di un assistente utile solo nei propri prodotti e non su larga scala. Inoltre, per ogni assistente, è richiesto un \textit{proof of concept} atto a dimostrare le funzionalità più significative che sono state riscontrate.\\
Sulla base dei risultati ottenuti nel corso dell'analisi si concretizza la seconda parte del lavoro: creare un'applicazione con un'interfaccia vocale, progettata secondo i canoni studiati in precedenza dagli altri assistenti e basata su grammatiche costruite mediante la tecnologia di Zucchetti, che permetta di intrattenere una conversazione con l'utente rispondendo alle sue richieste. \\
Il contenuto dell'applicazione non è vincolato dall'azienda in quanto l'obiettivo è approfondire l'abilità conversazionale e non costruire un'applicazione direttamente utilizzata nel software Zucchetti. In comune accordo è stato scelto il contesto della data di nascita poiché risulta ricca di sfumature sia nella richieste che nelle risposte e possiede parametri specifici e funzionali alla sperimentazione sulla capacità di conversazione.
Lo scopo generale del progetto di stage è quindi svolgere un'attività di ricerca e di approfondimento nella tematica del linguaggio naturale poiché è una tecnologia non ancora pienamente affermata ma con ampi margini di sviluppo.

\section{Obiettivi}
	\subsection{Classificazione}
	La classificazione degli obiettivi permette di identificarli univocamente e rispetta la seguente notazione:
	\begin{itemize}
		\item \textit{OB-x}: obiettivi obbligatori, vincolanti e primario richiesti dall'azienda;
		\item \textit{OD-x}: obiettivi desiderabili, non vincolanti o strettamente necessari ma dal riconoscibile valore aggiunto;
		\item \textit{OF-x}: obiettivi facoltativi dal valore aggiunto riconoscibile ma difficilmente raggiunbili a causa di agenti esterni all'attività di stage.
	\end{itemize}
	Il simbolo 'x' rappresenta un numero progressivo intero maggiore di zero.
	
	\subsection{Obiettivi obbligatori}
	\begin{itemize}
		\item \textit{\underline{OB-1}}: studio e analisi delle capacità e delle modalità di utilizzo lato sviluppatore di Assistant;
		\item \textit{\underline{OB-2}}: implementazione sottoforma di \textit{proof of concept} di una \textit{Action} basata su una funzionalità particolarmente rilevante di Assistant, scoperta durante la ricerca;
		\item \textit{\underline{OB-3}}: studio e analisi delle capacità e delle modalità di utilizzo lato sviluppatore di Alexa;
		\item \textit{\underline{OB-4}}: implementazione sottoforma di \textit{proof of concept} di una \textit{Skill} basata su una funzionalità particolarmente rilevante di Alexa, scoperta durante la ricerca;
		\item \textit{\underline{OB-5}}: redazione di un documento che tratti l'analisi dettagliata e comparativa in merito alle tecnologie degli assistenti virtuali studiati;
		\item \textit{\underline{OB-6}}: studio di regole e caratteristiche dell'algoritmo \textit{HPP (Hidden Probabilistic Parser)} di Zucchetti per la costruzione di grammatiche che riconoscono il linguaggio naturale;
		\item \textit{\underline{OB-7}}: realizzazione di un'applicazione basata su una grammatica generata da \textit{HPP} con interfaccia vocale e capacità conversazionale.
	\end{itemize}
	\subsection{Obiettivi desiderabili}
	Inizialmente era stato identificato come obiettivo desiderabile l'implementazione di una componente di addestramento per l'applicazione finale. Tuttavia nel corso dello stage, date anche alcune interessanti funzionalità trovate, questo obiettivo è stato modificato nel seguente:
	\begin{itemize}
		\item \textit{\underline{OD-1}}: implementazione di un meccanismo che permetta lo scambio di informazioni specifiche e possibilmente memorizzate durante la conversazione.
	\end{itemize}
	\subsection{Obiettivi facoltativi}
	\begin{itemize}
		\item \textit{\underline{OF-1}}: studio e analisi delle capacità e delle modalità di utilizzo lato sviluppatore di Siri;
		\item \textit{\underline{OF-2}}: implementazione sottoforma di \textit{proof of concept} di una \textit{Skill} basata su una funzionalità particolarmente rilevante di Siri, scoperta durante la ricerca.
	\end{itemize} 

\section{Pianificazione delle attività}
Lo stage è stato svolto in modalità \textit{smart working} con la possibilità di rimanere in comunicazione con il tutor aziendale attraverso Skype. In particolare l'orario di lavoro è stato 9:00-13:00, 14:00-18:00 ed sono stati stabiliti sia una chiamata al termine di ogni giornata che la compilazione di un registro delle attività giornaliero per tracciare e monitorare il lavoro svolto oltre a ricevere feedback costanti dal tutor. \\
La pianificazione è stata svolta su un totale di 320 ore suddivise secondo quanto riportato nella tabella seguente:
\\ \\
\begin{tabularx}{\textwidth}{|c|c|X|}
	\hline
	\textbf{Durata in ore} & \textbf{Date (inizio - fine)} & \textbf{Attività} \\\hline
	
	40 & 04/05/2020 - 08/05/2020 & Studio di Assistant e implementazione di un proof of concept. \\
	\hline
	40 & 11/05/2020 - 15/05/2020 & Studio di Alexa e implementazione di un proof of concept. \\
	\hline
	40 & 18/05/2020 - 22/05/2020 & Studio di Siri e implementazione di un proof of concept. \\
	\hline
	40 & 25/05/2020 - 29/05/2020 & Test e documentazione comparativa di quanto svolto nelle settimane precedenti. \\
	\hline
	40 & 01/06/2020 - 05/06/2020 & Apprendimento di sistema creato da Zucchetti per l'interazione mediante comandi vocali. \\
	\hline
	40 & 08/06/2020 - 12/06/2020 & Realizzazione di un'applicazione che implementi una grammatica costruita utilizzando la tecnologia di Zucchetti. \\
	\hline
	40 & 15/06/2020 - 19/06/2020 & Implementazione del meccanismo per lo scambo di informazioni con capacità di memorizzazione. \\
	\hline
	40 & 22/06/2020 - 26/06/2020 & Test e documentazione di quanto svolto nelle settimane precedenti. \\	
	\hline
\end{tabularx}
\\ \\
La pianificazione di dettaglio di ogni settimana è stata strutturata sulla base degli obiettivi stabiliti e viene ora riportata.
	\subsection*{Prima Settimana}
	Gli obiettivi prefissati sono:
	\begin{itemize}
		\item \textit{\underline{OB-1}};
		\item \textit{\underline{OB-2}}.
	\end{itemize}
	Le attività previste sono l'analisi di Assistant sotto alcuni aspetti principali:
	\begin{itemize}
		\item modalità di implemetazione per gli intenti relativi ad \textit{Action} di ogni tipologia;
		\item progettazione del prodotto software soprattutto all'interfaccia vocale;
		\item interazione dell'utente con l'assistente;
		\item modalità e sicurezza nel trasferimento dei dati;
		\item sviluppo di un \textit{proof of concept} che implementi una \textit{Action} per dare concretezza allo studio fatto.
	\end{itemize}
	\subsection*{Seconda Settimana}
	Gli obiettivi prefissati sono:
	\begin{itemize}
		\item \textit{\underline{OB-3}};
		\item \textit{\underline{OB-4}}.
	\end{itemize}
	Le attività previste sono l'analisi di Alexa sotto alcuni aspetti principali:
	\begin{itemize}
		\item modalità di implemetazione per gli intenti relativi ad \textit{Skill} di ogni tipologia;
		\item progettazione del prodotto software soprattutto all'interfaccia vocale;
		\item interazione dell'utente con l'assistente;
		\item modalità e sicurezza nel trasferimento dei dati;
		\item sviluppo di un \textit{proof of concept} che implementi una \textit{Skill} per dare concretezza allo studio fatto.
	\end{itemize}
	\subsection*{Terza Settimana}
	Gli obiettivi prefissati sono:
	\begin{itemize}
		\item \textit{\underline{OF-1}};
		\item \textit{\underline{OF-2}}.
	\end{itemize}
	Le attività previste sono l'analisi di Siri sotto alcuni aspetti principali:
	\begin{itemize}
		\item modalità di implemetazione per gli intenti relativi ad \textit{Skill} di ogni tipologia;
		\item progettazione del prodotto software soprattutto all'interfaccia vocale;
		\item interazione dell'utente con l'assistente;
		\item modalità e sicurezza nel trasferimento dei dati;
		\item sviluppo di un \textit{proof of concept} che implementi una \textit{Skill} per dare concretezza allo studio fatto 
	\end{itemize}
	\subsection*{Quarta Settimana}
	L'obiettivo prefissato è:
	\begin{itemize}
		\item \textit{\underline{OB-5}}.
	\end{itemize}
	Le attività previste sono:
	\begin{itemize}
		\item verifica del funzionamento dei \textit{proof of concept} realizzati;
		\item stesura della documentazione atta a riportare in modo dettagliato la comparazione tra le tecnologie studiate nelle settimane precedenti.
	\end{itemize}
	\subsection*{Quinta Settimana}
	L'obiettivo prefissato è:
	\begin{itemize}
		\item \textit{\underline{OB-6}}.
	\end{itemize}
	Le attività previste sono:
	\begin{itemize}
		\item studio dell'algoritmo \textit{HPP} di Zucchetti;
		\item esecuzione di alcune prove concrete per verificarne il corretto apprendimento.
	\end{itemize}
	\subsection*{Sesta Settimana}
	L'obiettivo prefissato è:
	\begin{itemize}
		\item \textit{\underline{OB-7}}.
	\end{itemize}
	Le attività previste sono:
	\begin{itemize}
		\item progettazione di un'applicazione, anch'essa sotto forma di \textit{proof of concept}, che implementi un'interfaccia vocale per il riconoscimento della data di nascita.
		\item realizzazione dell'applicazione.
	\end{itemize}
	\subsection*{Settima Settimana}
	L'obiettivo prefissato è:
	\begin{itemize}
		\item \textit{\underline{OD-01}}.
	\end{itemize}
	Le attività previste sono:
	\begin{itemize}
		\item analisi del meccanismo che permette lo scambio di informazioni con memorizzazione sulla base di quanto già implementato negli assistenti virtuali studiati;
		\item implementazione del meccanismo.
	\end{itemize}
	\subsection*{Ottava Settimana}
	In questo ultimo periodo è prevista l'implementazione di eventuali test e la stesura della documentazione finale sull'applicazione realizzata.



%vincoli metodologici ?
%Strumenti utilizzati
%Aspettative