% !TEX encoding = UTF-8
% !TEX TS-program = pdflatex
% !TEX root = ../tesi.tex

%**************************************************************
\chapter{Lo stage}
\label{cap:lo-stage}
%**************************************************************

%\intro{Brevissima introduzione al capitolo}\\

%**************************************************************
\section{Descrizione del progetto}
Il progetto di stage è legato ad uno degli ambiti più innovativi della ricerca scientifica e tecnologica: il riconoscimento del linguaggio naturale.
Nel corso degli ultimi anni l'azienda sta affrontando la nuova sfida della comprensione di comandi vocali integrata nei propri prodotti e ha deciso di puntare sulle capacità di giovani studenti universitari per esplorare ancor più questo tema. È tuttora in via di sviluppo una tecnologia che permette la creazione di grammatiche attraverso regole ben precise ed in merito a questo argomento è stato costruito lo stage.\\
La proposta di Zucchetti è di analizzare in modo dettagliato e comparativo i tre assistenti virtuali più utilizzati e, sulla carta, migliori:
\begin{itemize}
	\item Assistant: assistente di Google presente in quasi tutti i dispositivi Android oltre ad essere integrato nella linea Google Home e Nest;
	\item Alexa: assistente di Amazon presente principalemente nella linea di dispositivi Echo ma integrabile nei dispositivi con sistema operativo Android e IOS;
	\item Siri: assistente di Apple presente esclusivamente in tutti i prodotti Apple.
\end{itemize}
Lo studio deve comprendere tutte le capacità che posseggono, ad eccezione di quelle relative alla costruzione di smart home in quanto non di diretto interesse per i prodotti aziendali, mentre di particolare interesse è la costruzione di \textit{Skill} personalizzate ed integrabili nei propri progetti. \\
Sulla base dei risultati riscontrati emerge l'idea concreta proposta dall'azienda per lo stage: creare un'applicazione con un'interfaccia vocale, progettata secondo i canoni studiati in precedenza dagli altri assistenti e basata su grammatiche costruite tramite la tecnologia di Zucchetti, che permetta di intrattenere una conversazione con l'utente rispondendo alle sue richieste.

%Descrizione
%Obiettivi
%Pianificazione attività
%Strumenti utilizzati
%Aspettative