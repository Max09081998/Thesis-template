% !TEX encoding = UTF-8
% !TEX TS-program = pdflatex
% !TEX root = ../tesi.tex

%**************************************************************
\chapter{Lo stage}
\label{cap:lo-stage}
%**************************************************************

%\intro{Brevissima introduzione al capitolo}\\

%**************************************************************
\section{Descrizione del progetto}
Il progetto di stage è legato ad uno degli ambiti più innovativi della ricerca scientifica e informatica: il riconoscimento e l'elaborazione del linguaggio naturale da parte di un calcolatore. \\
Nel corso degli ultimi anni, l'azienda sta affrontando la sfida della comprensione di comandi vocali per permette agli utenti di comunicare con i propri prodotti. È infatti in via di sviluppo una loro tecnologia in grado di creare grammatiche atte a riconoscere il linguaggio naturale mediante regole ben precise. In merito a questo argomento si è costruito il progetto di stage articolato poi in due parti:
\begin{enumerate}
	\item studio degli assistenti virtuali più comuni ed utilizzati;
	\item creazione di un'applicazione sotto forma di \textit{proof of concept} capace di intrattenere una conversazione.
\end{enumerate}
La prima parte richiede un'analisi dettagliata e comparativa dei tre assistenti virtuali più utilizzati e, secondo diversi studi, migliori:
\begin{itemize}
	\item Assistant: assistente di Google presente in quasi tutti i dispositivi Android e nei prodotti della linea Google Home e Nest;
	\item Alexa: assistente di Amazon presente in tutta la linea di dispositivi Echo ed integrabile anche nei dispositivi con sistema operativo Android e IOS;
	\item Siri: assistente di Apple presente esclusivamente in tutti i prodotti Apple.
\end{itemize}
Lo studio deve comprendere tutte le capacità che posseggono, con particolare attenzione alla costruzione di \textit{Skill} personalizzate ed integrabili nei propri progetti, ad eccezione di quelle relative alla costruzione di \textit{smart home} in quanto non sono di diretto interesse per i prodotti aziendali; l'azienda infatti punta alla realizzazione di un assistente utile solo nei propri prodotti e non su larga scala. Inoltre, per ogni assistente, è richiesto un \textit{proof of concept} atto a dimostrare le funzionalità più significative che sono state riscontrate.\\
Sulla base dei risultati ottenuti nel corso dell'analisi si concretizza la seconda parte del lavoro: creare un'applicazione con un'interfaccia vocale, progettata secondo i canoni studiati in precedenza dagli altri assistenti e basata su grammatiche costruite mediante la tecnologia di Zucchetti, che permetta di intrattenere una conversazione con l'utente rispondendo alle sue richieste. \\
Il contenuto dell'applicazione non è vincolato dall'azienda in quanto l'obiettivo è approfondire l'abilità conversazionale e non costruire un'applicazione direttamente utilizzata nel software Zucchetti. In comune accordo è stato scelto il contesto della data di nascita poiché risulta ricca di sfumature sia nella richieste che nelle risposte e possiede parametri specifici e funzionali alla sperimentazione sulla capacità di conversazione.
Lo scopo generale del progetto di stage è quindi svolgere un'attività di ricerca e di approfondimento nella tematica del linguaggio naturale poiché è una tecnologia non ancora pienamente affermata ma con ampi margini di sviluppo.

\section{Obiettivi}
\subsection{Classificazione}
La classificazione degli obiettivi permette di identificarli univocamente e rispetta la seguente notazione:
\begin{itemize}
	\item \textit{OB-x}: obiettivi obbligatori, vincolanti e primario richiesti dall'azienda;
	\item \textit{OD-x}: obiettivi desiderabili, non vincolanti o strettamente necessari ma dal riconoscibile valore aggiunto;
	\item \textit{OF-x}: obiettivi facoltativi dal valore aggiunto riconoscibile ma difficilmente raggiunbili a causa di agenti esterni all'attività di stage.
\end{itemize}
Il simbolo 'x' rappresenta un numero progressivo intero maggiore di zero.

\subsection{Obiettivi obbligatori}
\begin{itemize}
	\item \textit{\underline{OB-1}}: studio e analisi delle capacità e delle modalità di utilizzo lato sviluppatore di Assistant;
	\item \textit{\underline{OB-2}}: implementazione sottoforma di \textit{proof of concept} di una \textit{Action} basata su una funzionalità particolarmente rilevante di Assistant, scoperta durante la ricerca;
	\item \textit{\underline{OB-3}}: studio e analisi delle capacità e delle modalità di utilizzo lato sviluppatore di Alexa;
	\item \textit{\underline{OB-4}}: implementazione sottoforma di \textit{proof of concept} di una \textit{Skill} basata su una funzionalità particolarmente rilevante di Alexa, scoperta durante la ricerca;
	\item \textit{\underline{OB-5}}: redazione di un documento che tratti l'analisi dettagliata e comparativa in merito alle tecnologie degli assistenti virtuali studiati;
	\item \textit{\underline{OB-6}}: studio di regole e caratteristiche dell'algoritmo \textit{HPP (Hidden Probabilistic Parser)} di Zucchetti per la costruzione di grammatiche che riconoscono il linguaggio naturale;
	\item \textit{\underline{OB-7}}: realizzazione di un'applicazione basata su una grammatica generata da \textit{HPP} con interfaccia vocale e capacità conversazionale.
\end{itemize}
\subsection{Obiettivi desiderabili} 
\begin{itemize}
	\item \textit{\underline{OD-1}}: implementazione di un meccanismo che permetta lo scambio di informazioni specifiche e possibilmente memorizzate durante la conversazione.
\end{itemize}
\subsection{Obiettivi facoltativi}
\begin{itemize}
	\item \textit{\underline{OF-1}}: studio e analisi delle capacità e delle modalità di utilizzo lato sviluppatore di Siri;
	\item \textit{\underline{OF-2}}: implementazione sottoforma di \textit{proof of concept} di una \textit{Skill} basata su una funzionalità particolarmente rilevante di Siri, scoperta durante la ricerca.
\end{itemize} 


\section{Pianificazione delle attività}
%Strumenti utilizzati
%Aspettative