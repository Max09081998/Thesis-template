% !TEX encoding = UTF-8
% !TEX TS-program = pdflatex
% !TEX root = ../tesi.tex

%**************************************************************
\chapter{Introduzione}
\label{cap:introduzione}
%**************************************************************

%Introduzione al contesto applicativo.\\

%\noindent Esempio di utilizzo di un termine nel glossario \\
%\gls{api}. \\

%\noindent Esempio di citazione in linea \\
%\cite{site:agile-manifesto}. \\

%\noindent Esempio di citazione nel pie' di pagina \\
%citazione\footcite{womak:lean-thinking} \\

%**************************************************************
\section{Zucchetti S.p.A}
Zucchetti è un'azienda italiana fondata più di 40 anni fa che produce soluzioni software, hardware e servizi per soddisfare le esigenze tecnologiche dei propri clienti, anche a livello internazionale. Le sedi sono dislocate in numerose città italiane, tra cui Padova dove ho svolto lo stage e Lodi in cui risiede la sede amministrativa.


\begin{figure}[htbp]
	\begin{center}
		\includegraphics[height=3cm]{ZUCCHETTI_logo-new.jpg}
	\end{center}
\end{figure}

Domenico Zucchetti, fondatore dell'azienda, ha avuto la geniale intuizione di costruire un software per agevolare il lavoro dei commercialisti, allora completamente cartaceo e manuale. Con il passare degli anni il suo prodotto ha avuto un successo sempre maggiore tanto da ottenere collaborazioni con aziende del calibro di IBM. A partire da questo, Zucchetti ha continuato a perseguire l'innovazione del proprio prodotto integrandolo con nuovi moduli quali ERP e la più recente fatturazione elettronica, con l'obiettivo di conferire maggiore flessibilità e adattabilità per ogni tipologia di impresa, senza più limitarsi alla categoria dei commercialisti. Forte di questo, Zucchetti si è espansa a livello nazionale ed internazionale ed ora si pone sul mercato con una vasta gamma di servizi per numerosi settori quali industrie manifatturiere, trasporti, logistica, sanità, fitness e molti altri. \\

\section{Lo stage proposto}
Uno dei pilastri di Zucchetti che ne ha caratterizzato la costante crescita è l'innovazione e la propensione alla ricerca di nuove tecnologie. \\
A questo proposito la sede di Padova è caratterizzata da un reparto di ricerca e sviluppo nel quale sono stato inserito durante la mia esperienza di stage. Il coordinatore di questo reparto è il dott. Gregorio Piccoli che mi ha proposto il progetto di tirocinio e si è offerto come mio tutor. \\
L'azienda sta cercando di introdurre nei propri prodotti, in particolare nel software gestionale, un'interfaccia vocale che permetta agli utenti un'interazione più veloce e spontanea per le operazioni più comuni. L'obiettivo è quindi implementare un'interfaccia diversa da quella grafica, con caratteristiche proprie, che esprima un nuovo modo di comunicare con le applicazioni, ancora poco sviluppato ma dalle grandi potenzialità. \\
Il principale ostacolo da superare per il raggiungimento di tale obiettivo è il riconoscimento del linguaggio naturale attraverso un'applicativo software. Per fare ciò l'azienda ha sviluppato delle regole per la creazione di grammatiche in grado di riconoscere comandi espressi con il linguaggio naturale. \\
Tuttavia è una tecnologia ancora in fase di sviluppo ed in merito a ciò, come progetto di stage, mi sono stati proposte due tematiche pensate per suddividere lo stage in due parti:
\begin{enumerate}
	\item analizzare i tre assistenti virtuali attualmente più diffusi sul mercato, \emph{Assistant}\glsfirstoccur, \emph{Alexa}\glsfirstoccur e \emph{Siri}\glsfirstoccur, per comprenderne le abilità e, qualora esistesse la possibilità, permettere agli utenti di utilizzarli per impartire comandi al proprio software gestionale Zucchetti;
	\item esplorare la possibilità di aggiungere la capacità conversazionale in un'applicazione che utilizzi una grammatica costruita con la nuova tecnologia dell'azienda, ispirandosi anche agli assistenti virtuali precedentemente studiati.
\end{enumerate}

%**************************************************************
\section{Organizzazione del testo}

\begin{description}
    \item[{\hyperref[cap:processi-metodologie]{Il secondo capitolo}}] descrive ...
    
    \item[{\hyperref[cap:descrizione-stage]{Il terzo capitolo}}] approfondisce ...
    
    \item[{\hyperref[cap:analisi-requisiti]{Il quarto capitolo}}] approfondisce ...
    
    \item[{\hyperref[cap:progettazione-codifica]{Il quinto capitolo}}] approfondisce ...
    
    \item[{\hyperref[cap:verifica-validazione]{Il sesto capitolo}}] approfondisce ...
    
    \item[{\hyperref[cap:conclusioni]{Nel settimo capitolo}}] descrive ...
\end{description}

Riguardo la stesura del testo, relativamente al documento sono state adottate le seguenti convenzioni tipografiche:
\begin{itemize}
	\item gli acronimi, le abbreviazioni e i termini ambigui o di uso non comune menzionati vengono definiti nel glossario, situato alla fine del presente documento;
	\item per la prima occorrenza dei termini riportati nel glossario viene utilizzata la seguente nomenclatura: \emph{parola}\glsfirstoccur;
	\item i termini in lingua straniera o facenti parti del gergo tecnico sono evidenziati con il carattere \emph{corsivo}.
\end{itemize}