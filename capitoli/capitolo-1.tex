% !TEX encoding = UTF-8
% !TEX TS-program = pdflatex
% !TEX root = ../tesi.tex

%**************************************************************
\chapter{Introduzione}
\label{cap:introduzione}
%**************************************************************

%Introduzione al contesto applicativo.\\

%\noindent Esempio di utilizzo di un termine nel glossario \\
%\gls{api}. \\

%\noindent Esempio di citazione in linea \\
%\cite{site:agile-manifesto}. \\

%\noindent Esempio di citazione nel pie' di pagina \\
%citazione\footcite{womak:lean-thinking} \\

%**************************************************************
\section{Zucchetti S.p.A}
Zucchetti è un'azienda italiana fondata più di 40 anni fa che produce soluzioni software, hardware e servizi per soddisfare le esigenze tecnologiche dei propri clienti, anche a livello internazionale. Le sedi sono dislocate in numerose città italiane, tra cui Padova, mentre la sede principale ed amministrativa è a Lodi.


\begin{figure}[htbp]
	\begin{center}
		\includegraphics[height=6cm]{ZUCCHETTI_logo-new.jpg}
	\end{center}
\end{figure}

Domenico Zucchetti, fondatore dell'azienda, ha avuto la geniale intuizione di costruire un software per agevolare il lavoro dei commercialisti, allora completamente cartaceo e manuale, e con il passare degli anni questo suo prodotto ha avuto un successo sempre maggiore. A partire da questo Zucchetti ha continuato nell'innovazione del proprio prodotto, integrandolo con nuovi moduli quali ERP e la più recente fatturazione elettronica, con l'obiettivo di conferire flessibilità e adattabilità ad ogni tipologia di impresa. Forte di questo, Zucchetti si è espansa a livello nazionale ed internazionale ed ora si pone sul mercato con una vasta gamma di servizi.

\section{Lo stage proposto}
Il punto di forza di questa azienda è la continua ricerca dell'innovazione su tutti i settori di propria competenza. A questo proposito la sede di Padova è composta da un reparto di ricerca e sviluppo nel quale sono stato inserito durante la mia esperienza.

Introduzione all'idea dello stage.

%**************************************************************
\section{Tecnologie e strumenti utilizzati}


%**************************************************************
\section{Organizzazione del testo}

\begin{description}
    \item[{\hyperref[cap:processi-metodologie]{Il secondo capitolo}}] descrive ...
    
    \item[{\hyperref[cap:descrizione-stage]{Il terzo capitolo}}] approfondisce ...
    
    \item[{\hyperref[cap:analisi-requisiti]{Il quarto capitolo}}] approfondisce ...
    
    \item[{\hyperref[cap:progettazione-codifica]{Il quinto capitolo}}] approfondisce ...
    
    \item[{\hyperref[cap:verifica-validazione]{Il sesto capitolo}}] approfondisce ...
    
    \item[{\hyperref[cap:conclusioni]{Nel settimo capitolo}}] descrive ...
\end{description}

Riguardo la stesura del testo, relativamente al documento sono state adottate le seguenti convenzioni tipografiche:
\begin{itemize}
	\item gli acronimi, le abbreviazioni e i termini ambigui o di uso non comune menzionati vengono definiti nel glossario, situato alla fine del presente documento;
	\item per la prima occorrenza dei termini riportati nel glossario viene utilizzata la seguente nomenclatura: \emph{parola}\glsfirstoccur;
	\item i termini in lingua straniera o facenti parti del gergo tecnico sono evidenziati con il carattere \emph{corsivo}.
\end{itemize}