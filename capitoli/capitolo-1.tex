% !TEX encoding = UTF-8
% !TEX TS-program = pdflatex
% !TEX root = ../tesi.tex

%**************************************************************
\chapter{Introduzione}
\label{cap:introduzione}
%**************************************************************

%Introduzione al contesto applicativo.\\

%\noindent Esempio di utilizzo di un termine nel glossario \\
%\gls{api}. \\

%\noindent Esempio di citazione in linea \\
%\cite{site:agile-manifesto}. \\

%\noindent Esempio di citazione nel pie' di pagina \\
%citazione\footcite{womak:lean-thinking} \\

%**************************************************************
\section{Zucchetti S.p.A}
Zucchetti è un'azienda italiana fondata più di 40 anni fa che produce soluzioni software, hardware e servizi per soddisfare le esigenze tecnologiche dei propri clienti, anche a livello internazionale. Le sedi sono dislocate in numerose città italiane, tra cui Padova dove ho svolto lo stage e Lodi in cui risiede la sede amministrativa.


\begin{figure}[htbp]
	\begin{center}
		\includegraphics[height=3cm]{ZUCCHETTI_logo-new.jpg}
	\end{center}
\end{figure}

Domenico Zucchetti, fondatore dell'azienda, ha avuto la geniale intuizione di costruire un software per agevolare il lavoro dei commercialisti, allora completamente cartaceo e manuale. Con il passare degli anni il suo prodotto ha avuto un successo sempre maggiore tanto da ottenere collaborazioni con aziende del calibro di \emph{\gls{ibmg}}\glsfirstoccur. A partire da questo, Zucchetti ha continuato a perseguire l'innovazione del proprio prodotto integrandolo con nuovi moduli quali \emph{\gls{erpg}}\glsfirstoccur e la più recente fatturazione elettronica, con l'obiettivo di conferire maggiore flessibilità e adattabilità per ogni tipologia di impresa, senza limitarsi ai commercialisti. Forte di questo, Zucchetti si è espansa a livello nazionale ed internazionale ponendosi sul mercato con una vasta gamma di servizi per numerosi settori quali industrie manifatturiere, trasporti, logistica, sanità, fitness e molti altri. \\

\section{Lo stage proposto}
Uno dei pilastri di Zucchetti che ha contruibuto alla sua espansione è l'innovazione e la propensione alla ricerca di nuove tecnologie. \\
A questo proposito la sede di Padova è composta da un reparto di ricerca e sviluppo nel quale sono stato inserito durante la mia esperienza di stage. Il coordinatore di tale reparto è il dott. Gregorio Piccoli che mi ha proposto il progetto e offrendosi come mio tutor. \\
L'azienda sta cercando di introdurre nei propri prodotti, in particolare nel software gestionale, un'interfaccia vocale che permetta agli utenti di interagire in modo più veloce e spontaneo nelle operazioni comuni. L'obiettivo è quindi implementare un'interfaccia diversa da quella grafica, con caratteristiche proprie, che esprima un nuovo modo di comunicare con le applicazioni; seppur ancora poco sviluppato ha dalle grandi potenzialità. \\
Per raggiungere tale obiettivo il principale ostacolo da superare è il riconoscimento del linguaggio naturale attraverso un'applicativo software. Perciò l'azienda ha sviluppato delle regole per la creazione di grammatiche capaci di riconoscere comandi vocali. Tuttavia è una tecnologia ancora in fase di sviluppo ed in merito a ciò mi sono stati proposte due tematiche come progetto di stage, pensate per suddividere il lavoro in due parti:
\begin{enumerate}
	\item analizzare i tre assistenti virtuali attualmente più diffusi sul mercato, Assistant, Alexa e Siri, per comprenderne le abilità e verificare la possibilità di utilizzarli per permettere agli utenti comunicare con il software gestionale dell'azienda;
	\item esplorare la possibilità di implementare la capacità conversazionale in una \emph{\gls{nlug}}\glsfirstoccur che utilizzi una grammatica costruita con la nuova tecnologia dell'azienda, ispirandosi anche agli assistenti virtuali precedentemente studiati.
\end{enumerate}

%**************************************************************
\section{Organizzazione del testo}

\begin{description}
    \item[{\hyperref[cap:processi-metodologie]{Il secondo capitolo}}] descrive ...
    
    \item[{\hyperref[cap:descrizione-stage]{Il terzo capitolo}}] approfondisce ...
    
    %\item[{\hyperref[cap:analisi-requisiti]{Il quarto capitolo}}] approfondisce ...
    
    %\item[{\hyperref[cap:progettazione-codifica]{Il quinto capitolo}}] approfondisce ...
    
    %\item[{\hyperref[cap:verifica-validazione]{Il sesto capitolo}}] approfondisce ...
    
    %\item[{\hyperref[cap:conclusioni]{Nel settimo capitolo}}] descrive ...
\end{description}

Riguardo la stesura del testo, relativamente al documento sono state adottate le seguenti convenzioni tipografiche:
\begin{itemize}
	\item gli acronimi, le abbreviazioni e i termini ambigui o di uso non comune menzionati vengono definiti nel glossario, situato alla fine del presente documento;
	\item per la prima occorrenza dei termini riportati nel glossario viene utilizzata la seguente nomenclatura: \emph{parola}\glsfirstoccur;
	\item i termini in lingua straniera o facenti parti del gergo tecnico sono evidenziati con il carattere \emph{corsivo}.
\end{itemize}