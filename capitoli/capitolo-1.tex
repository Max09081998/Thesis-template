% !TEX encoding = UTF-8
% !TEX TS-program = pdflatex
% !TEX root = ../tesi.tex

%**************************************************************
\chapter{Introduzione}
\label{cap:introduzione}
%**************************************************************

%Introduzione al contesto applicativo.\\

%\noindent Esempio di utilizzo di un termine nel glossario \\
%\gls{api}. \\

%\noindent Esempio di citazione in linea \\
%\cite{site:agile-manifesto}. \\

%\noindent Esempio di citazione nel pie' di pagina \\
%citazione\footcite{womak:lean-thinking} \\

%**************************************************************
\section{Descrizione del progetto}
Il progetto di stage è legato ad uno dei rami più conosciuti dell'intelligenza artificiale: l'interpretazione del linguaggio naturale. Questo argomento non è trattato nel percorso di studi della laurea triennale e personalmente lo ritengo una sfida ed una motivazione aggiuntiva per conoscere e scoprire nuove tecnologie. \\
Nelle prime settimane il lavoro si è articolato in attività di ricerca, sperimentazione e documentazione su tre assistenti virtuali: Assistant, Alexa e Siri. Successivamente ho studiato la tecnologia Zucchetti per l'interpretazione del linguaggio naturale e ho realizzato un'applicazione basata su di essa, che rispetti i principi di progettazione appresi durante le precedenti ricerche ed implementi un meccanismo di conversazione.
%**************************************************************
\section{Zucchetti S.p.A}
Il progetto di stage è stato svolto con Zucchetti, un'azienda italiana fondata più di 40 anni fa che produce soluzioni software, hardware e servizi per soddisfare le esigenze tecnologiche dei propri clienti, anche a livello internazionale. Le sedi sono dislocate in numerose città italiane tra cui Padova dove ho svolto lo stage e Lodi che rappresenta la sede amministrativa.

\begin{figure}[htbp]
	\begin{center}
		\includegraphics[height=3cm]{ZUCCHETTI_logo-new.jpg}
		\caption{Logo di Zucchetti}
	\end{center}
\end{figure}

Domenico Zucchetti, fondatore dell'azienda, ha avuto la geniale intuizione di costruire un software per agevolare il lavoro dei commercialisti, allora completamente cartaceo e manuale. Con il passare degli anni il suo prodotto ha riscontrato sempre maggior successo tanto da ottenere collaborazioni con aziende del calibro di \emph{\gls{ibmg}}\glsfirstoccur. A partire da questo, Zucchetti ha continuato a perseguire la strada dell'innovazione integrando il proprio prodotto con moduli nuovi, quali \emph{\gls{erpg}}\glsfirstoccur e la più recente fatturazione elettronica, per conferire maggiore flessibilità e adattabilità ad ogni tipologia di impresa, senza limitarsi ai commercialisti. \\
Forte del suo prodotto, negli ultimi decenni l'azienda si è espansa a livello nazionale ed internazionale, ponendosi sul mercato con una vasta gamma di servizi per innumerevoli settori tra cui industria manifatturiera, trasporti, logistica, sanità, fitness e molti altri. \\
Uno dei pilastri di Zucchetti che ha indubbiamente contribuito alla sua espansione è l'innovazione e la propensione alla ricerca di nuove tecnologie. A questo proposito la sede di Padova è composta da un reparto di ricerca e sviluppo, nel quale sono stato inserito durante la mia esperienza di stage, coordinato dal dott. Gregorio Piccoli che mi ha proposto il progetto e si è offerto come mio tutor.
%**************************************************************
\section{Lo stage proposto}
\subsection{Contesto del lavoro}
L'azienda sta cercando di introdurre nei propri prodotti, in particolare nel software gestionale, un'interfaccia vocale che permetta agli utenti di interagire in modo più veloce e spontaneo nelle operazioni comuni. Il loro obiettivo è quindi implementare un'interfaccia diversa da quella grafica, con caratteristiche proprie, che esprima un modo nuovo di comunicare con le applicazioni. Esso, seppur ancora poco sviluppato, possiede grandi potenzialità. \\
Per raggiungere questo traguardo il principale ostacolo da superare è il riconoscimento del linguaggio naturale attraverso un'applicativo software. Perciò l'azienda ha sviluppato delle regole per la creazione di \emph{grammatiche} che permettano di eseguire il \emph{parsing} dei comandi vocali. Tuttavia è una tecnologia ancora in fase di sviluppo e proprio per questo, come progetto di stage, mi sono state proposte due tematiche mirate ad un'attività di esplorazione. Esse sono pensate per suddividere il lavoro in due parti:
\begin{enumerate}
	\item analizzare i tre assistenti virtuali attualmente più diffusi sul mercato, Assistant, Alexa e Siri, per comprenderne le abilità e verificare i loro possibili impieghi nei prodotti dell'azienda;
	\item implementare una \emph{\gls{nlug}}\glsfirstoccur basata su una \emph{\gls{gramg}}\glsfirstoccur costruita con la nuova tecnologia Zucchetti, possibilmente con capacità conversazionale, ed ispirata agli assistenti virtuali precedentemente analizzati.
\end{enumerate}
Più nello specifico gli obiettivi da perseguire sono descritti nella sezione \hyperref[section:obiettivi]{2.2} ma possono essere riassunti nei seguenti punti:
\begin{itemize}
	\item analizzare le tecnologie offerte ed i principi di implementazione delle abilità di Assistant, Alexa e Siri con \emph{\gls{pocg}}\glsfirstoccur che le concretizzano;
	\item capire dalle ricerche svolte se esistono componenti rilevanti per i progetti aziendali che riguardano la realizzazione di un'interfaccia vocale;
	\item implementare un'applicazione ispirata alle nozioni studiate e che utilizzi una \emph{\gls{gramg}} costruita mediante la tecnologia Zucchetti, possibilmente con capacità conversazionale.
\end{itemize}
\subsection{Problematiche incontrate}
Lo stage è stato svolto esclusivamente da remoto e ciò ha in parte influenzato il piano di lavoro, ponendo come facoltativi degli obiettivi quasi sicuramente raggiungibili in presenza come quelli dell'attività legata a Siri. Infatti, per la maggior parte dei compiti, necessita di un computer con sistema operativo MacOS che ho potuto reperire solo per un breve periodo di tempo ed inoltre, a causa delle restrizioni dell'account sviluppatore Apple a disposizione, non si è potuta concludere l'attività. \\
Ad ogni modo, considerando la nuova esperienza di lavoro remoto sia per me che per l'azienda, lo stage è stato molto positivo e non sono emerse ulteriori problematiche.
\subsection{Sintesi dei risultati}
Nonostante il problema legato all'attività con Siri, è stato possibile concludere lo stage rispettando il piano di lavoro e con risultati ottimi. \\
Gli esiti della ricerca delle prime settimane hanno soddisfatto le aspettative, confermando alcune intuizioni del tutor e facendo emergere nuovi elementi importanti per gli sviluppi della loro interfaccia vocale. Le conferme riguardano l'utilizzo di Assistant ed Alexa per comunicare con i prodotti aziendali attraverso appositi \emph{\gls{sdkg}}\glsfirstoccur offerti rispettivamente da Google e Amazon; Apple invece non fornisce lo stesso per Siri affermando la sua politica di ecosistema chiuso. I nuovi elementi rilevati sono le tecniche di progettazione e realizzazione della capacità conversazionale e l'integrazione dell'assistente virtuale all'interno di applicazioni mobile, fornita solo da Assistant e Siri. Essi sono emersi principalmente grazie ai \emph{\gls{pocg}} costruiti in modo mirato sulle funzionalità più interessanti per l'azienda e hanno fornito molti spunti di riflessione su possibili nuove implementazioni. I loro risultati sono così riassunti:
\begin{itemize}
	\item \emph{\gls{pocg}} di Assistant: utilizzo dell'assistente virtuale per attivare determinate funzionalità di un'applicazione. Nel mio caso si tratta di un timer per fare attività fisica;
	\item \emph{\gls{pocg}} di Alexa: effettuare una conversazione con l'assistente virtuale finalizzata ad eseguire una determinata funzione. Nel mio caso si tratta di riferire la data di nascita al mio applicativo;
	\item \emph{\gls{pocg}} di Siri: utilizzo dell'assistente virtuale per creare comandi personalizzati di attivazione delle applicazioni o delle loro specifiche funzionalità. Sebbene non sia stato possibile completarlo con successo, i dati ricavati sono stati ugualmente soddisfacenti.
\end{itemize}
Anche l'esito dell'applicazione ha dato ottimi frutti, confermando le caratteristiche delle \emph{grammatiche} Zucchetti ovvero facilità e grande flessibilità nell'utilizzo e portando alla luce nozioni e tematiche rilevanti per gli sviluppi della loro \emph{\gls{nlug}}, quali capacità di memorizzazione, comprensione del contesto e architettura del software che ne fa uso. L'applicazione è costituita da una \emph{\gls{nlug}} capace di interpretare data di nascita e di compleanno attraverso una conversazione. Questo significa, ad esempio, che l'utente può fornire una data parziale ed in seguito l'applicazione chiederà i dati mancanti con domande mirate.
%**************************************************************
\section{Organizzazione del testo}
Il capitolo finora trattato è di introduzione mentre il seguito del documento è organizzato secondo questa struttura:
\begin{description}
    \item[{\hyperref[cap:lo-stage]{Il secondo capitolo}}] descrive obiettivi, pianificazione, strumenti e tecnologie del progetto di stage ed infine le motivazioni che mi hanno portato a sceglierlo.
    
    \item[{\hyperref[cap:assistenti-virtuali]{Il terzo capitolo}}] descrive il lavoro di analisi e ricerca sugli assistenti virtuali Assistant, Alexa e Siri e le considerazioni tratte al termine.
    
    \item[{\hyperref[cap:applicazione]{Il quarto capitolo}}] descrive inizialmente la tecnologia Zucchetti ed in seguito l'applicazione costruita per la comprensione del linguaggio naturale con capacità conversazionale.
    
    \item[{\hyperref[cap:conclusione]{Il quinto capitolo}}] rappresenta le conclusioni dell'elaborato: riassume il lavoro svolto, il raggiungimento degli obiettivi ed infine riporta un'analisi retrospettiva dell'intera esperienza di stage.
\end{description}

Riguardo la stesura del testo di questo documento sono state adottate le seguenti convenzioni tipografiche:
\begin{itemize}
	\item gli acronimi, le abbreviazioni e i termini ambigui o di uso non comune menzionati vengono definiti nel glossario, situato alla fine del presente documento;
	\item per la prima occorrenza dei termini riportati nel glossario viene utilizzata la seguente nomenclatura: \emph{parola}\glsfirstoccur;
	\item i termini in lingua straniera o facenti parti del gergo tecnico sono evidenziati con il carattere \emph{corsivo}.
\end{itemize}