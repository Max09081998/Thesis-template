% !TEX encoding = UTF-8
% !TEX TS-program = pdflatex
% !TEX root = ../tesi.tex

%**************************************************************
\chapter{Introduzione}
\label{cap:introduzione}
%**************************************************************

%Introduzione al contesto applicativo.\\

%\noindent Esempio di utilizzo di un termine nel glossario \\
%\gls{api}. \\

%\noindent Esempio di citazione in linea \\
%\cite{site:agile-manifesto}. \\

%\noindent Esempio di citazione nel pie' di pagina \\
%citazione\footcite{womak:lean-thinking} \\

%**************************************************************
\section{Zucchetti S.p.A}
Zucchetti è un'azienda italiana fondata più di 40 anni fa che produce soluzioni software, hardware e servizi per soddisfare le esigenze tecnologiche dei propri clienti, anche a livello internazionale. Le sedi sono dislocate in numerose città italiane, tra cui Padova dove ho svolto lo stage e Lodi in cui risiede la sede amministrativa.


\begin{figure}[htbp]
	\begin{center}
		\includegraphics[height=4cm]{ZUCCHETTI_logo-new.jpg}
	\end{center}
\end{figure}

Domenico Zucchetti, fondatore dell'azienda, ha avuto la geniale intuizione di costruire un software per agevolare il lavoro dei commercialisti, allora completamente cartaceo e manuale, e con il passare degli anni questo suo prodotto ha avuto un successo sempre maggiore. A partire da questo Zucchetti ha continuato nell'innovazione del proprio prodotto, integrandolo con nuovi moduli quali ERP e la più recente fatturazione elettronica, con l'obiettivo di conferire flessibilità e adattabilità ad ogni tipologia di impresa. Forte di questo, Zucchetti si è espansa a livello nazionale ed internazionale ed ora si pone sul mercato con una vasta gamma di servizi.

\section{Lo stage proposto}
Uno degli elementi cardine dell'azienda è la continua ricerca mirata all'innovazione su tutti i settori di propria competenza. A questo proposito la sede di Padova è caratterizzata da un reparto di ricerca e sviluppo nel quale sono stato inserito durante la mia esperienza. Il coordinatore di questo reparto è il dott. Gregorio Piccoli che mi ha proposto il progetto di stage. \\
L'azienda sta cercando di introdurre nei propri prodotti, in particolar modo nel gestionale, un'interfaccia vocale per permettere agli utenti un'interazione più veloce e spontaea. L'obiettivo è quindi implementare una nuova interfaccia, diversa da quella grafica, con caratteristiche proprie che permettano un'interazione nuova, ancora poco sviluppata ma dalle grosse potenzialità. \\
Il principale ostacolo da superare è il riconoscimento del linugaggio naturale attraverso un'applicativo software. Per fare ciò ha sviluppato delle regole per la creazione di grammatiche in grado di riconoscere comandi espressi con il linguaggio naturale. \\
In merito a questa nuova tecnologia sono emersi due ambiti di ricerca che mi sono stati proposti come progetto di stage e sono i seguenti:
\begin{itemize}
	\item analizzare i tre assistenti virtuali attualmente più diffusi sul mercato, Assistant, Alexa e Siri, per comprenderne le abilità e, qualora esistesse la possibilità, permettere agli utenti di comunicare con il proprio software gestionale Zucchetti attraverso di essi;
	\item esplorare la possibilità di aggiungere la capacità conversazionale in un'applicazione basata su una grammatica costruita secondo le regole definite dall'azienda.
\end{itemize}

In particolare 

%**************************************************************
\section{Tecnologie e strumenti utilizzati}


%**************************************************************
\section{Organizzazione del testo}

\begin{description}
    \item[{\hyperref[cap:processi-metodologie]{Il secondo capitolo}}] descrive ...
    
    \item[{\hyperref[cap:descrizione-stage]{Il terzo capitolo}}] approfondisce ...
    
    \item[{\hyperref[cap:analisi-requisiti]{Il quarto capitolo}}] approfondisce ...
    
    \item[{\hyperref[cap:progettazione-codifica]{Il quinto capitolo}}] approfondisce ...
    
    \item[{\hyperref[cap:verifica-validazione]{Il sesto capitolo}}] approfondisce ...
    
    \item[{\hyperref[cap:conclusioni]{Nel settimo capitolo}}] descrive ...
\end{description}

Riguardo la stesura del testo, relativamente al documento sono state adottate le seguenti convenzioni tipografiche:
\begin{itemize}
	\item gli acronimi, le abbreviazioni e i termini ambigui o di uso non comune menzionati vengono definiti nel glossario, situato alla fine del presente documento;
	\item per la prima occorrenza dei termini riportati nel glossario viene utilizzata la seguente nomenclatura: \emph{parola}\glsfirstoccur;
	\item i termini in lingua straniera o facenti parti del gergo tecnico sono evidenziati con il carattere \emph{corsivo}.
\end{itemize}