% !TEX encoding = UTF-8
% !TEX TS-program = pdflatex
% !TEX root = ../tesi.tex

%**************************************************************
% Sommario
%**************************************************************
\cleardoublepage
\phantomsection
\pdfbookmark{Sommario}{Sommario}
\begingroup
\let\clearpage\relax
\let\cleardoublepage\relax
\let\cleardoublepage\relax

\chapter*{Sommario}

Il presente documento descrive il lavoro svolto durante il periodo di stage, della durata di 320 ore, dal laureando Massimo Toffoletto presso l'azienda Zucchetti S.p.A.
Gli obiettivi da raggiungere sono stati molteplici e suddivisi in due parti.\\
Nella prima parte ho studiato il funzionamento dei tre principali assistenti virtuali nel seguente ordine: Assistant, Alexa e Siri. Gli obiettivi sono stati quindi l'analisi dei singoli assistenti e lo svolgimento di una comparazione tra gli stessi, sia delle funzionalità offerte agli sviluppatori che delle capacità riconoscitive del linguaggio naturale. A questo proposito ho realizzato un proof of concept dimostrativo per ciascuno di essi. \\
Nella seconda parte gli obiettivi sono stati lo studio di regole per la realizzazione di grammatiche che interpretano il linguaggio naturale, implementate da Zucchetti ma ancora in via di sviluppo, e la costruzione di un'applicazione che le utilizzi. Infine ho implementato un'interfaccia vocale con capacità conversazionale di interagire che si integri con la tecnologia Zucchetti.

%\vfill
%
%\selectlanguage{english}
%\pdfbookmark{Abstract}{Abstract}
%\chapter*{Abstract}
%
%\selectlanguage{italian}

\endgroup			

\vfill

