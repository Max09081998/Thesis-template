
%**************************************************************
% Acronimi
%**************************************************************
\renewcommand{\acronymname}{Acronimi e abbreviazioni}

\newacronym[description={\glslink{apig}{Application Program Interface}}]
{api}{API}{Application Program Interface}

\newacronym[description={\glslink{umlg}{Unified Modeling Language}}]
{uml}{UML}{Unified Modeling Language}

\newacronym[description={\glslink{erpg}{Enterprise Resource Planning}}]
{erp}{ERP}{Enterprise Resource Planning}

\newacronym[description={\glslink{ibmg}{International Business Machines Corporation}}]
{ibm}{IBM}{International Business Machines Corporation}

%**************************************************************
% Glossario
%**************************************************************
\renewcommand{\glossaryname}{Glossario}

\newglossaryentry{apig}
{
	name=\glslink{api}{API},
	text=Application Program Interface,
	sort=api,
	description={in informatica con il termine \emph{Application Programming Interface API} (ing. interfaccia di programmazione di un'applicazione) si indica ogni insieme di procedure disponibili al programmatore, di solito raggruppate a formare un set di strumenti specifici per l'espletamento di un determinato compito all'interno di un certo programma. La finalità è ottenere un'astrazione, di solito tra l'hardware e il programmatore o tra software a basso e quello ad alto livello semplificando così il lavoro di programmazione}
}

\newglossaryentry{umlg}
{
	name=\glslink{uml}{UML},
	text= Unified Modeling Language (UML),
	sort=uml,
	description={in ingegneria del software \emph{UML, Unified Modeling Language} (ing. linguaggio di modellazione unificato) è un linguaggio di modellazione e specifica basato sul paradigma object-oriented. L'\emph{UML} svolge un'importantissima funzione di ``lingua franca'' nella comunità della progettazione e programmazione a oggetti. Gran parte della letteratura di settore usa tale linguaggio per descrivere soluzioni analitiche e progettuali in modo sintetico e comprensibile a un vasto pubblico}
}
\newglossaryentry{erpg}
{
    name=\glslink{erp}{ERP},
    text=Enterprise Resource Planning,
    sort=erp,
    description={in informatica con il termine \emph{ERP, Enterprise Resource Planning} (ing. pianificazione delle risorse d'impresa) è un software di gestione che integra tutti i processi di business rilevanti di un'azienda e tutte le sue funzioni quali vendite, acquisti, gestione magazzino, finanza e contabilità. Integra quindi tutte le attività aziendali in un unico sistema che risulta essere indispensabile per supportare il Management.}
}

\newglossaryentry{ibmg}
{
	name=\glslink{ibm}{IBM},
	text=International Business Machines Corporation,
	sort=ibm,
	description={L'\emph{IBM, International Business Machines Corporation} è un'azienda statunitense, la più antica e tra le maggiori al mondo nel settore informatico. Produce e commercializza hardware, software per computer, middleware e servizi informatici, offrendo anche infrastrutture, servizi di hosting, cloud computing, intelligenza artificiale e consulenza nel settore informatico e strategico.}
}

