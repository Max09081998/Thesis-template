
%**************************************************************
% Acronimi
%**************************************************************
\renewcommand{\acronymname}{Acronimi e abbreviazioni}

\newacronym[description={\glslink{apig}{Application Program Interface}}]
{api}{API}{Application Program Interface}

\newacronym[description={\glslink{erpg}{Enterprise Resource Planning}}]
{erp}{ERP}{Enterprise Resource Planning}

\newacronym[description={\glslink{ibmg}{International Business Machines Corporation}}]
{ibm}{IBM}{International Business Machines Corporation}

\newacronym[description={\glslink{nlug}{Natural Language Understanding}}]
{nlu}{NLU}{Natural Language Understanding}

\newacronym[description={\glslink{pocg}{Proof of Concept}}]
{poc}{PoC}{Proof of Concept}

\newacronym[description={\glslink{jvmg}{Java Virtual Machine}}]
{jvm}{JVM}{Java Virtual Machine}

\newacronym[description={\glslink{es6}{ECMAScript6}}]
{es6}{ES6}{ECMAScript6}

\newacronym[description={\glslink{html}{HyperText Markup Language}}]
{html}{HTML}{}

\newacronym[description={\glslink{css}{Cascading Style Sheets}}]
{css}{CSS}{}

\newacronym[description={\glslink{php}{Hypertext Preprocessor}}]
{php}{PHP}{}

\newacronym[description={\glslink{ram}{Random Access Memory}}]
{ram}{RAM}{}

\newacronym[description={\glslink{sdk}{Software Development Kit}}]
{sdk}{SDK}{}

\newacronym[description={\glslink{http}{HyperText Transfer Protocol}}]
{http}{HTTP}{}
%**************************************************************
% Glossario
%**************************************************************
\renewcommand{\glossaryname}{Glossario}

\newglossaryentry{apig}
{
	name=\glslink{api}{API},
	text=API,
	sort=api,
	description={in informatica con il termine \emph{Application Programming Interface API} (ing. interfaccia di programmazione di un'applicazione) si indica ogni insieme di procedure disponibili al programmatore, di solito raggruppate a formare un set di strumenti specifici per l'espletamento di un determinato compito all'interno di un certo programma. La finalità è ottenere un'astrazione, di solito tra l'hardware e il programmatore o tra software a basso e quello ad alto livello semplificando così il lavoro di programmazione}
}


\newglossaryentry{erpg}
{
    name=\glslink{erp}{ERP},
    text=ERP,
    sort=erp,
    description={in informatica con il termine \emph{ERP, Enterprise Resource Planning} (ing. pianificazione delle risorse d'impresa) è un software di gestione che integra tutti i processi di business rilevanti di un'azienda e tutte le sue funzioni quali vendite, acquisti, gestione magazzino, finanza e contabilità. Integra quindi tutte le attività aziendali in un unico sistema che risulta essere indispensabile per supportare il Management}
}

\newglossaryentry{ibmg}
{
	name=\glslink{ibm}{IBM},
	text=IBM,
	sort=ibm,
	description={L'\emph{IBM, International Business Machines Corporation} è un'azienda statunitense, la più antica e tra le maggiori al mondo nel settore informatico. Produce e commercializza hardware, software per computer, middleware e servizi informatici, offrendo anche infrastrutture, servizi di hosting, cloud computing, intelligenza artificiale e consulenza nel settore informatico e strategico}
}

\newglossaryentry{nlug}
{
	name=\glslink{nlu}{NLU},
	text=NLU,
	sort=ibm,
	description={La \emph{NLU, Natural Language Understanding} (ing. comprensione del linguaggio naturale) è un ramo dell'elaborazione del linguaggio naturale nell'intelligenza artificiale}
}

\newglossaryentry{pocg}
{
	name=\glslink{pocg}{PoC},
	text=PoC,
	sort=poc,
	description={Con \emph{PoC, Proof of Concept} (ing. prova di fattibilità) si intende una realizzazione incompleta o abbozzata di un determinato progetto o metodo, allo scopo di provarne la fattibilità o dimostrare la fondatezza di alcuni principi o concetti costituenti. Un esempio tipico è quello di un prototipo}
}

\newglossaryentry{jvmg}
{
	name=\glslink{jvmg}{JVM},
	text=JVM,
	sort=jvm,
	description={La \emph{JVM, Java Virtual Machine} è il componente della piattaforma Java che esegue i programmi tradotti in bytecode dopo una prima fase di compilazione. Alcuni dei linguaggi di programmazione che possono essere tradotti in bytecode sono Java, Kotlin e Scala}
}

\newglossaryentry{ttg}
{
	name=\glslink{ttg}{Test di Turing},
	text=Test di Turing,
	sort=Test di Turing,
	description={Il \emph{Test di Turing} è un criterio costruito dallo scienziato Alan Turing nel 1950 per determinare se una macchina sia in grado di comprendere il linguaggio naturale e più in generale di pensare}
}

\newglossaryentry{sdkg}
{
	name=\glslink{sdkg}{Software Development Kit},
	text=Software Development Kit,
	sort=Software Development Kit,
	description={Un \emph{SDK, Software Development Kit} (ing. pacchetto di sviluppo software) in informatica, indica un insieme di strumenti per lo sviluppo e la documentazione di software}
}

\newglossaryentry{httpg}
{
	name=\glslink{httpg}{HyperText Transfer Protocol},
	text=HTTP,
	sort=HTTP,
	description={In informatica \emph{HTTP, HyperText Transfer Protocol} (ing. protocollo di trasferimento di un ipertesto) è un protocollo a livello applicativo usato come principale sistema per la trasmissione d'informazioni sul Web. Esiste una versione detta \emph{HTTPS, HyperText Transfer Protocol over Secure Socket Layer} che implementa le stesse funzionalità applicando uno strato di crittografia}
}

\newglossaryentry{buildg}
{
	name=\glslink{buildg}{build},
	text=Build,
	sort=build,
	description={Nello sviluppo del software, \emph{build} indica il processo di trasformazione del codice sorgente in un artefatto eseguibile}
}

\newglossaryentry{firebaseg}
{
	name=\glslink{firebaseg}{Firebase},
	text=Firebase,
	sort=Firebase,
	description={\emph{Firebase} è una piattaforma di sviluppo per applicazioni Web e mobile sviluppata da Firebase Inc. nel 2011 e acquisita da Google nel 2014}
}

\newglossaryentry{authg}
{
	name=\glslink{firebaseg}{OAuth 2.0},
	text=OAuth 2.0,
	sort=OAuth 2.0,
	description={\emph{OAuth 2.0} è un protocollo di rete aperto e standard, progettato specificamente per lavorare con il protocollo \emph{HTTP}. Consente l'emissione di un token d'accesso da parte di un server che fornisce autorizzatizioni ad un client previa approvazione dell'utente proprietario della risorsa cui si intende accedere. Rispetto alla sua versione precedente presenta una chiara divisione dei ruoli, implementando un mediatore tra client e server}
}

\newglossaryentry{rldg}
{
	name=\glslink{rldg}{railroad},
	text=Railroad,
	sort=railroad,
	description={I \emph{diagrammi railroad}, detti anche \emph{diagrammi di sintassi} consistono in una rapprensentazione grafica per grammatiche libere da contesto}
}

\newglossaryentry{parsg}
{
	name=\glslink{parsg}{parser},
	text=Parser,
	sort=parser,
	description={In informatica il \emph{parser} è un software che realizza il parsing ovvero un processo che implementa l'analisi sintattica. Più in dettaglio analizza un flusso continuo di dati in input e determina la correttezza della sua struttura grazie ad una data grammatica formale}
}

\newglossaryentry{gramg}
{
	name=\glslink{gramg}{grammatica},
	text=Grammatica,
	sort=grammatica,
	description={Nella teoria dei linguaggi formali una \emph{grammatica} (detta anche grammatica formale) è una struttura astratta che descrive un linguaggio (formale) in modo preciso. È definita anche come sistema di regole che delineano matematicamente un insieme potenzilamente infinito di sequenze finite di simboli appartenenti ad un alfabeto finito}
}

\newglossaryentry{cfg}
{
	name=\glslink{cfg}{grammatica libera da contesto},
	text=Grammatica libera da contesto,
	sort=grammatica libera da contesto,
	description={Una \emph{grammatica libera da contesto} è una struttura astratta che descrive un linguaggio (formale) in modo preciso con una notazione naturalmente ricorsiva}
}

\newglossaryentry{brainstorming}
{
	name=\glslink{brainstorming}{brainstorming},
	text=Brainstorming,
	sort=brainstorming,
	description={Il \emph{brainstorming} è un metodo decisionale in cui la ricerca della soluzione di un dato problema è effettuata mediante sedute intensive di dibattito e confronto delle idee espresse liberamente dai partecipanti}
}
