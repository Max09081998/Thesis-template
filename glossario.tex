
%**************************************************************
% Acronimi
%**************************************************************
\renewcommand{\acronymname}{Acronimi e abbreviazioni}

\newacronym[description={\glslink{apig}{Application Program Interface}}]
{api}{API}{Application Program Interface}

\newacronym[description={\glslink{erpg}{Enterprise Resource Planning}}]
{erp}{ERP}{Enterprise Resource Planning}

\newacronym[description={\glslink{ibmg}{International Business Machines Corporation}}]
{ibm}{IBM}{International Business Machines Corporation}

\newacronym[description={\glslink{nlug}{Natural Language Understanding}}]
{nlu}{NLU}{Natural Language Understanding}
%**************************************************************
% Glossario
%**************************************************************
\renewcommand{\glossaryname}{Glossario}

\newglossaryentry{apig}
{
	name=\glslink{api}{API},
	text=Application Program Interface,
	sort=api,
	description={in informatica con il termine \emph{Application Programming Interface API} (ing. interfaccia di programmazione di un'applicazione) si indica ogni insieme di procedure disponibili al programmatore, di solito raggruppate a formare un set di strumenti specifici per l'espletamento di un determinato compito all'interno di un certo programma. La finalità è ottenere un'astrazione, di solito tra l'hardware e il programmatore o tra software a basso e quello ad alto livello semplificando così il lavoro di programmazione}
}


\newglossaryentry{erpg}
{
    name=\glslink{erp}{ERP},
    text=ERP,
    sort=erp,
    description={in informatica con il termine \emph{ERP, Enterprise Resource Planning} (ing. pianificazione delle risorse d'impresa) è un software di gestione che integra tutti i processi di business rilevanti di un'azienda e tutte le sue funzioni quali vendite, acquisti, gestione magazzino, finanza e contabilità. Integra quindi tutte le attività aziendali in un unico sistema che risulta essere indispensabile per supportare il Management.}
}

\newglossaryentry{ibmg}
{
	name=\glslink{ibm}{IBM},
	text=IBM,
	sort=ibm,
	description={L'\emph{IBM, International Business Machines Corporation} è un'azienda statunitense, la più antica e tra le maggiori al mondo nel settore informatico. Produce e commercializza hardware, software per computer, middleware e servizi informatici, offrendo anche infrastrutture, servizi di hosting, cloud computing, intelligenza artificiale e consulenza nel settore informatico e strategico.}
}

\newglossaryentry{nlug}
{
	name=\glslink{nlu}{NLU},
	text=NLU,
	sort=ibm,
	description={La \emph{NLU, Natural Language Understanding} (ing. comprensione del linguaggio naturale) è un ramo dell'elaborazione del linguaggio naturale nell'intelligenza artificiale.}
}

